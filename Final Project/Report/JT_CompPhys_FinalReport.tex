\documentclass{article}

% Essentials and Formatting
\usepackage[utf8]{inputenc}                          % encodes unicode stuff
\usepackage{graphicx}                                % helps make pretty pictures
\usepackage{multicol}                                % multicolumn/multirow cells in tables
\usepackage{dcolumn}                                 % Align table columns on decimal point
\usepackage[table,usenames,dvipsnames]{xcolor}       % extra colors (i like ForestGreen)
\usepackage{xparse}                                  % make new commands i think
\usepackage[margin=1in]{geometry}                    % controls document shape/size
%\usepackage[backend=biber, sorting=nyt]{biblatex-chicago}              % for bibliographies
\usepackage{calligra}                                % solely for Griffiths r
\usepackage{fancyhdr}                                % control of headers and footers
\usepackage{lastpage}                                % not entirely sure tbh
\usepackage{titlesec}                                % control of section heading formatting
\usepackage{listings}                                % for formatting code nicely
\usepackage{url}                                     % for url in appendix

% Math and Science
\usepackage{bbm, bm}                   % "blackboard bold" for real numbers
\usepackage{amsmath}                   % makes typing math stuff easier
\usepackage{amsthm}                    % makes typing actual math stuff easier
\usepackage{siunitx}                   % makes typing units easier
\usepackage{derivative}                % makes typing derivatives easier
\usepackage[version=4]{mhchem}         % makes typing chemistry easier

% Graphs, Tables, and Figures
\usepackage{tikz}                      % pictures
\usepackage{pgfplots}                  % pretty graphs
\usepackage{pgfplotstable}             % formatting options for tables
\usepackage{adjustbox}                 % puts a box around things that are too big and makes them smaller
\usepackage{caption}                   % caption stuff for tables and plots
\usepackage{subcaption}                % same but for subtables and subfigures
\usepackage{csvsimple}                 % easy compatability with comma separated value files

% Citations and references
\usepackage{natbib}                                  % for STEM, uses bibtex NOT biber
\setcitestyle{square, super}
%\usepackage[backend=biber, sorting=nyt]{biblatex}   % for basically only Music History
\usepackage{bookmark}                                % bookmarks. includes hyperref
\usepackage{cleveref}                                % automatically determines the kind of reference

% Package configuration
\hypersetup{colorlinks=true, allcolors=blue}
\sisetup{per-mode = fraction, list-final-separator = {, and }, group-digits=integer, fraction-function=\tfrac}
\pgfplotsset{compat=1.18}
%\addbibresource{sources.bib} % Remember to run biber 'filename' in the terminal if using
%\bibliography{sources} % Move to end of file when using natbib
\fancypagestyle{firststyle}
{
    \fancyhf{}
    \fancyhead[L]{PHYS 4023 $-$ Final Report}
    \fancyhead[R]{Joseph Temple}
    \renewcommand{\headrulewidth}{0pt} % removes horizontal header line
    \setlength{\headheight}{26.0pt}

    \fancyfoot[C]{\thepage}
}

% The actual document
\begin{document}

\begin{titlepage}
    \begin{center}
        \vspace*{1cm}
 
        \huge
        \textbf{Final Project}
 
        \Large
        Active Brownian Particles in an External Field
             

        \vfill

        \textbf{Joseph Temple}
        
 
        \vfill

        Arkansas Tech University \\
        PHYS 4023 Computational Physics \\
        December 12, 2025  
    \end{center}
 \end{titlepage}


\pagestyle{firststyle}

\section{Introduction}

Active Brownian Particles (ABPs) are a minimal model for self-propelled agents in a thermal environment. 
The three key ideas are:
\begin{itemize}
    \item \textbf{Brownian particles:} random motion due to collisions with a surrounding fluid (thermal noise).
    \item \textbf{Active:} particles also possess an intrinsic self-propulsion at a characteristic speed.
    \item \textbf{External field:} particles respond to spatial variations in an environmental potential.
\end{itemize}

This work implements a simple discrete-time ABP model and examines how an external potential modifies 
the single-particle dynamics and the collective emergent behavior.

\section{Background: Brownian and Active Motion}

\subsection{Regular Brownian Motion}
Brownian motion refers to the random translational and rotational perturbations that particles experience 
due to collisions with many much lighter fluid molecules. In the overdamped limit we assume that collisions 
rapidly randomize the orientation of a particle's velocity vector while leaving its speed (magnitude of 
self-propulsion) approximately constant.

\subsection{Active Brownian Motion}
Active Brownian Particles add a self-propelled velocity of fixed magnitude \(v\) whose direction evolves 
under rotational diffusion. In 2D we denote by \(\theta_i(t)\) the orientation angle of particle \(i\) at 
time \(t\). The orientation is driven by stochastic noise and the position is updated by deterministic 
self-propulsion plus any external forces.

\section{Model and Discrete-Time Updates}

The discrete-time update rules used in the simulations are:

\begin{align}
\theta_i(t+\Delta t) &= \theta_i(t) + \sqrt{2D\,\Delta t}\,\xi_i(t), \label{eq:theta-update}\\
x_i(t+\Delta t) &= x_i(t) + \Big[ v\cos\theta_i(t) \;-\; \mu\,\nabla_x V\big(x_i(t),y_i(t)\big) \Big]\Delta t, \label{eq:x-update}\\
y_i(t+\Delta t) &= y_i(t) + \Big[ v\sin\theta_i(t) \;-\; \mu\,\nabla_y V\big(x_i(t),y_i(t)\big) \Big]\Delta t. \label{eq:y-update}
\end{align}

Notes on the parameters and terms:
\begin{itemize}
    \item \(\xi_i(t)\) is a sample from a standard normal distribution \(\mathcal{N}(0,1)\); 
    samples across different \(t\) and \(i\) are independent.
    \item \(D\) is the rotational diffusion coefficient. \(D\) can be related to temperature \(T\), 
    viscosity of the fluid \(\eta\), and particle radius \(R\) through 
    \(D \sim \dfrac{k_B T}{6\pi \eta R}\). $2D\Delta t$ is the variance of the probability
    distribution defining $\theta$ updates.
    \item \(v\) is the self-propulsion speed (kept constant).
    \item \(\mu\) is the motility, controlling how strongly the particle 
    responds to the potential gradient.
    \item \(-\nabla V\) gives the force field associated with potential \(V(x,y)\).
\end{itemize}

\section{External Potential}

To introduce nontrivial spatial dependence I used the following quartic potential, with four minima 
and one maximum:
\[
V(x,y) = A\bigl[x^4 - x^2 + y^4 - y^2 \bigr],
\]
where \(A\) sets the potential strength/scale. This choice produces multiple attractor regions and rich 
gradients that interact with the self-propulsion to produce nonlinear phase-like behaviour.

The corresponding force components are
\[
-\nabla_x V(x,y) = -A(4x^3 - 2x), \qquad -\nabla_y V(x,y) = -A(4y^3 - 2y).
\]

\section{Numerical Implementation}

Simulations were performed with the updates described in Eqs. \ref{eq:theta-update} $-$ \ref{eq:y-update} 
with a fixed timestep \(\Delta t\).  Typical choices used in experiments:
\begin{itemize}
    \item \(v \sim \mathcal{O}(1)\) (units set by simulation).
    \item \(\Delta t \ll 1\), chosen small enough to keep updates stable.
    \item \(A\) varied to explore weak-to-strong external fields.
    \item \(\mu\) scanned to vary sensitivity to the field (small \(\mu\) yields weak response; 
    large \(\mu\) yields strong response).
    \item \(D\) scanned to see effects of rotational diffusion (large \(D\) rapidly randomizes orientation; 
    small \(D\) preserves orientation longer).
\end{itemize}

We also implemented basic collision detection, both between particles and with the walls of the container.
In short, if the distance between the center of two particles becomes less than twice the radius
they intersect with one another, which we correct for by pushing them apart and reversing direction. A similar
rule is applied to collisions with the boundary, where a reflection along $x$ or $y$ is added when
the center of a particle gets within one radius of a vertical or horizontal boundary, respectively.

\section{Results and Observations}

With no external field (\(A=0\)) the particle performs persistent random walks whose 
persistence length depends on \(v/D\).
Adding the potential biases motion toward wells: when \(\mu A\) is moderate, particles tend 
    to be attracted to stable wells; with strong motility the interplay between self-propulsion and the 
    potential leads to trapping and escape events that depend sensitively on \(D\) and \(v\).

When many-particle or long-time behavior is considered, the system can exhibit emergent patterns not present 
in the single deterministic parts: multiple steady-state populations, metastable trapping regions, and 
transitions between regimes as parameters vary. This qualitatively matches the expectation that ABPs 
interacting with nontrivial fields produce phase-like transitions and rich stochastic dynamics.


We generated a number of short videos showing the behavior of the system in different regimes, notably
by varying $D$ and $\mu$. These animations and full code live on GitHub (see Appendix). As a dynamic phemenon,
an image embedded in a PDF does not provide a particularly strong impression of the work, but one
will be provided nonetheless.
\begin{figure}[ht]
    \centering

    \includegraphics[width=0.7\textwidth]{./fig/screenshot.png}

    \caption{Example of ABP visualization.}

    \label{fig:init}
\end{figure}

\section{Discussion and Conclusion}

The combination of self-propulsion, rotational diffusion, and spatially varying forces produces nontrivial 
steady states and dynamics. The quartic potential used here is simple but already generates a landscape with 
multiple stable/unstable regions; scanning \((D,\mu,A,v)\) surfaces reveals parameter ranges where trapping or 
persistent circulation dominate.

Active Brownian Particles in an external field demonstrate how simple stochastic microscopic rules 
combine to produce complex, nonlinear macroscopic behaviour. The minimal ABP model with a quartic 
separable potential reveals trapping, drift, and parameter-dependent transitions that are qualitatively 
representative of chemotactic or field-responsive active matter.

\section{Code Availability}
\label{sec:appendix}

The full code and animations mp4s are available on GitHub:
\begin{itemize}
    \item \url{https://github.com/josephtemple/JT_PHYS4023} (project root)
    \item Particularly relevant folders: \texttt{Final Project} and \texttt{Final\ Project/videos} 
    (see the repository for exported frames and movies).
\end{itemize}


\end{document}
